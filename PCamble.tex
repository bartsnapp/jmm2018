\graphicspath{
  {./}
  {1-1QuantitativeReasoning/}
  {1-2RelationsAndGraphs/}
  {1-3ChangingInTandem/}
  {2-1LinearEquations/}
  {2-2LinearModeling/}
  {2-3ExponentialModeling/}
  {3-1WhatIsAFunction/}
  {3-2FunctionProperties/}
  {3-3AverageRatesOfChange/}
  {3-4ExponentialFunctions/}
  {4-1BuildingNewFunctions/}
  {4-2Polynomials/}
  {4-3RationalFunctions/}
  {5-1Domain/}
  {5-2Range/}
  {5-3CompositionOfFunctions/}
  {6-1ZerosOfFunctions/}
  {6-2ZerosOfPolynomials/}
  {6-3ZerosOfFamousFunctions/}
  {7-1FunctionTransformations/}
  {8-1SystemsOfEquations/}
  {7-2FunctionTransformationsProject/}
  {1-1QuantitativeReasoning/exercises/}
  {1-2RelationsAndGraphs/exercises/}
  {../1-3ChangingInTandem/exercises/}
  {../2-1LinearEquations/exercises/}
  {../2-2LinearModeling/exercises/}
  {../2-3ExponentialModeling/exercises/}
  {../3-1WhatIsAFunction/exercises/}
  {../3-2FunctionProperties/exercises/}
  {../3-3AverageRatesOfChange/exercises/}
  {../3-4ExponentialFunctions/exercises/}
  {../4-1BuildingNewFunctions/exercises/}
  {../4-2Polynomials/exercises/}
  {../4-3RationalFunctions/exercises/}
  {../5-1Domain/exercises/}
  {../5-2Range/exercises/}
  {../5-3CompositionOfFunctions/exercises/}
  {../6-1ZerosOfFunctions/exercises/}
  {../6-2ZerosOfPolynomials/exercises/}
  {../6-3ZerosOfFamousFunctions/exercises/}
  {../7-1FunctionTransformations/exercises/}
  {../8-1SystemsOfEquations/exercises/}
  {../7-2FunctionTransformationsProject/exercises/}
}

\DeclareGraphicsExtensions{.pdf,.png,.jpg,.eps}


\newcommand{\mooculus}{\textsf{\textbf{MOOC}\textnormal{\textsf{ULUS}}}}

\usepackage[makeroom]{cancel} %% for strike outs

\ifweb
\else
\usepackage[most]{tcolorbox}
\fi


%\typeout{************************************************}
%\typeout{New Environments}
%\typeout{************************************************}

%% to fix for web can be removed when deployed offically with ximera2
\let\image\relax\let\endimage\relax
\NewEnviron{image}{% 
  \begin{center}\BODY\end{center}% center
}



\NewEnviron{folder}{
      \addcontentsline{toc}{section}{\textbf{\BODY}}
}


\ifweb
\let\summary\relax
\let\endsummary\relax
\newtheorem*{summary}{Summary}
\ximerizedEnvironment{summary}
\newtheorem*{callout}{Callout}
\ximerizedEnvironment{callout}
\newtheorem*{overview}{Overview}
\ximerizedEnvironment{overview}
\newtheorem*{objectives}{Objectives}
\ximerizedEnvironment{objectives}
\newtheorem*{motivatingQuestions}{Motivating Questions}
\ximerizedEnvironment{motivatingQuestions}
\newtheorem*{MM}{Metacognitive Moment}
\ximerizedEnvironment{MM}
\else
%% CALLOUT
\NewEnviron{callout}{
  \begin{tcolorbox}[colback=blue!5, breakable,pad at break*=1mm]
      \BODY
  \end{tcolorbox}
}
%% MOTIVATING QUESTIONS
\NewEnviron{motivatingQuestions}{
  \begin{tcolorbox}[ breakable,pad at break*=1mm]
    \textbf{\Large Motivating Questions}\hfill
    %\begin{itemize}[label=\textbullet]
      \BODY
    %\end{itemize}
  \end{tcolorbox}
}
%% OBJECTIVES
\NewEnviron{objectives}{  
    \vspace{.5in}
      %\begin{tcolorbox}[colback=orange!5, breakable,pad at break*=1mm]
    \textbf{\Large Learning Objectives}
    \begin{itemize}[label=\textbullet]
      \BODY
    \end{itemize}
    %\end{tcolorbox}
}
%% DEFINITION
\let\definition\relax
\let\enddefinition\relax
\NewEnviron{definition}{
  \begin{tcolorbox}[ breakable,pad at break*=1mm]
    \noindent\textbf{Definition}~
      \BODY
  \end{tcolorbox}
}
%% OVERVIEW
\let\overview\relax
\let\overview\relax
\NewEnviron{overview}{
  \begin{tcolorbox}[ breakable,pad at break*=1mm]
    \textbf{\Large Overview}
    %\begin{itemize}[label=\textbullet] %% breaks Xake
      \BODY
    %\end{itemize}
  \end{tcolorbox}
}
%% SUMMARY
\let\summary\relax
\let\endsummary\relax
\NewEnviron{summary}{
  \begin{tcolorbox}[ breakable,pad at break*=1mm]
    \textbf{\Large Summary}
    %\begin{itemize}[label=\textbullet] %% breaks Xake
      \BODY
    %\end{itemize}
  \end{tcolorbox}
}
%% REMARK
\let\remark\relax
\let\endremark\relax
\NewEnviron{remark}{
  \begin{tcolorbox}[colback=blue!5, breakable,pad at break*=1mm]
    \noindent\textbf{Remark}~
      \BODY
  \end{tcolorbox}
}
%% EXPLANATION
\let\explanation\relax
\let\endexplanation\relax
\NewEnviron{explanation}{
    \normalfont
    \noindent\textbf{Explanation}~
      \BODY
}
%% EXPLORATION
\let\exploration\relax
\let\endexploration\relax
\NewEnviron{exploration}{
  \begin{tcolorbox}[colback=yellow!10, breakable,pad at break*=1mm]
    \noindent\textbf{Exploration}~
      \BODY
  \end{tcolorbox}
}
%% METACOGNITIVE MOMENTS
\let\MM\relax
\let\endMM\relax
\NewEnviron{MM}{
  \begin{tcolorbox}[colback=pink!15, breakable,pad at break*=1mm]
    \noindent\textbf{Metacognitive Moment}~
      \BODY
  \end{tcolorbox}
}


\fi





%% %Notes on what envirnoment to use:  Example with Explanation in text; if they are supposed to answer- Problem; no answer - Exploration


%% %\typeout{************************************************}
%% %% Header and footers
%% %\typeout{************************************************}

%% \newcommand{\licenseAcknowledgement}{Licensed under Creative Commons 4.0}
%% \newcommand{\licenseAPC}{\renewcommand{\licenseAcknowledgement}{\textbf{Acknowledgements:} Active Prelude to Calculus (https://activecalculus.org/prelude) }}
%% \newcommand{\licenseSZ}{\renewcommand{\licenseAcknowledgement}{\textbf{Acknowledgements:} Stitz Zeager Open Source Mathematics (https://www.stitz-zeager.com/) }}
%% \newcommand{\licenseORCCA}{\renewcommand{\licenseAcknowledgement}{\textbf{Acknowledgements:}Original source material, products with readable and accessible
%% math content, and other information freely available at pcc.edu/orcca.}}
%% \newcommand{\licenseY}{\renewcommand{\licenseAcknowledgement}{\textbf{Acknowledgements:} Yoshiwara Books (https://yoshiwarabooks.org/)}}
%% \newcommand{\licenseOS}{\renewcommand{\licenseAcknowledgement}{\textbf{Acknowledgements:} OpenStax College Algebra (https://openstax.org/details/books/college-algebra)}}

%% \usepackage{fancyhdr}
%% \pagestyle{fancy}
%% \fancyhf{}
%% \fancyhead[R]{\sectionmark}
%% \fancyfoot[L]{\thepage}
%% \fancyfoot[C]{\licenseAcknowledgement}
%% \renewcommand{\headrulewidth}{0pt}
%% \renewcommand{\footrulewidth}{0pt}



%% %%%%%%%%%%%%%%%%%%%%%%%%%%%%%%%%%%%%%%%%%%%%%%
%% %%%%% THIS IS MY DEBUGGING LINE ERROR IS ABOVE
%% %%%%%%%%%%%%%%%%%%%%%%%%%%%%%%%%%%%%%%%%%%%%%%



%% %%%%%%%%%%%%%%%%



%% %\typeout{************************************************}
%% %\typeout{Table of Contents}
%% %\typeout{************************************************}


%% %% Edit this to change the font style
%% \newcommand{\sectionHeadStyle}{\sffamily\bfseries}


%% \makeatletter

%% %% part uses arabic numerals
%% \renewcommand*\thepart{\arabic{part}}



%% \renewcommand\chapterstyle{%
%%   \def\maketitle{%
%%     \addtocounter{titlenumber}{1}%
%%     \pagestyle{fancy}
%%     \phantomsection
%%     \addcontentsline{toc}{section}{\textbf{\thepart.\thetitlenumber\hspace{1em}\@title}}%
%%                     {\flushleft\small\sectionHeadStyle\@pretitle\par\vspace{-1.5em}}%
%%                     {\flushleft\LARGE\sectionHeadStyle\thepart.\thetitlenumber\hspace{1em}\@title \par }%
%%                     {\setcounter{problem}{0}\setcounter{sectiontitlenumber}{0}}%
%%                     \par}}





%% \renewcommand\sectionstyle{%
%%   \def\maketitle{%
%%     \addtocounter{sectiontitlenumber}{1}
%%     \pagestyle{fancy}
%%     \phantomsection
%%     \addcontentsline{toc}{subsection}{\thepart.\thetitlenumber.\thesectiontitlenumber\hspace{1em}\@title}%
%%     {\flushleft\small\sectionHeadStyle\@pretitle\par\vspace{-1.5em}}%
%%     {\flushleft\Large\sectionHeadStyle\thepart.\thetitlenumber.\thesectiontitlenumber\hspace{1em}\@title \par}%
%%     %{\setcounter{subsectiontitlenumber}{0}}%
%%     \par}}



%% \renewcommand\section{\@startsection{paragraph}{10}{\z@}%
%%                                      {-3.25ex\@plus -1ex \@minus -.2ex}%
%%                                      {1.5ex \@plus .2ex}%
%%                                      {\normalfont\large\sectionHeadStyle}}
%% \renewcommand\subsection{\@startsection{subparagraph}{10}{\z@}%
%%                                     {3.25ex \@plus1ex \@minus.2ex}%
%%                                     {-1em}%
%%                                     {\normalfont\normalsize\sectionHeadStyle}}



%% %% redefine Part
%% \renewcommand\part{%
%%    {\setcounter{titlenumber}{0}}
%%   \if@openright
%%     \cleardoublepage
%%   \else
%%     \clearpage
%%   \fi
%%   \thispagestyle{plain}%
%%   \if@twocolumn
%%     \onecolumn
%%     \@tempswatrue
%%   \else
%%     \@tempswafalse
%%   \fi
%%   \null\vfil
%%   \secdef\@part\@spart}

%% \def\@part[#1]#2{%
%%     \ifnum \c@secnumdepth >-2\relax
%%       \refstepcounter{part}%
%%       \addcontentsline{toc}{part}{\thepart\hspace{1em}#1}%
%%     \else
%%       \addcontentsline{toc}{part}{#1}%
%%     \fi
%%     \markboth{}{}%
%%     {\centering
%%      \interlinepenalty \@M
%%      \normalfont
%%      \ifnum \c@secnumdepth >-2\relax
%%        \huge\sffamily\bfseries \partname\nobreakspace\thepart
%%        \par
%%        \vskip 20\p@
%%      \fi
%%      \Huge \bfseries #2\par}%
%%     \@endpart}
%% \def\@spart#1{%
%%     {\centering
%%      \interlinepenalty \@M
%%      \normalfont
%%      \Huge \bfseries #1\par}%
%%     \@endpart}
%% \def\@endpart{\vfil\newpage
%%               \if@twoside
%%                \if@openright
%%                 \null
%%                 \thispagestyle{empty}%
%%                 \newpage
%%                \fi
%%               \fi
%%               \if@tempswa
%%                 \twocolumn
%%                 \fi}



%% \makeatother





%% %\typeout{************************************************}
%% %\typeout{Stuff from Ximera}
%% %\typeout{************************************************}



%% \usepackage{array}  %% This is for typesetting long division
%% \setlength{\extrarowheight}{+.1cm}
%% \newdimen\digitwidth
%% \settowidth\digitwidth{9}
%% \def\divrule#1#2{
%% \noalign{\moveright#1\digitwidth
%% \vbox{\hrule width#2\digitwidth}}}





%% \newcommand{\RR}{\mathbb R}
%% \newcommand{\R}{\mathbb R}
%% \newcommand{\N}{\mathbb N}
%% \newcommand{\Z}{\mathbb Z}

%% \newcommand{\sagemath}{\textsf{SageMath}}


%% %\renewcommand{\d}{\,d\!}
%% \renewcommand{\d}{\mathop{}\!d}
%% \newcommand{\dd}[2][]{\frac{\d #1}{\d #2}}
%% \newcommand{\pp}[2][]{\frac{\partial #1}{\partial #2}}
%% \renewcommand{\l}{\ell}
%% \newcommand{\ddx}{\frac{d}{\d x}}



%% %\newcommand{\unit}{\,\mathrm}
%% \newcommand{\unit}{\mathop{}\!\mathrm}
%% \newcommand{\eval}[1]{\bigg[ #1 \bigg]}
%% \newcommand{\seq}[1]{\left( #1 \right)}
%% \renewcommand{\epsilon}{\varepsilon}
%% \renewcommand{\phi}{\varphi}


%% \renewcommand{\iff}{\Leftrightarrow}

%% \DeclareMathOperator{\arccot}{arccot}
%% \DeclareMathOperator{\arcsec}{arcsec}
%% \DeclareMathOperator{\arccsc}{arccsc}
%% \DeclareMathOperator{\sign}{sign}


%% %\DeclareMathOperator{\divergence}{divergence}
%% %\DeclareMathOperator{\curl}[1]{\grad\cross #1}
%% \newcommand{\lto}{\mathop{\longrightarrow\,}\limits}

%% \renewcommand{\bar}{\overline}

%% \colorlet{textColor}{black}
%% \colorlet{background}{white}
%% \colorlet{penColor}{blue!50!black} % Color of a curve in a plot
%% \colorlet{penColor2}{red!50!black}% Color of a curve in a plot
%% \colorlet{penColor3}{red!50!blue} % Color of a curve in a plot
%% \colorlet{penColor4}{green!50!black} % Color of a curve in a plot
%% \colorlet{penColor5}{orange!80!black} % Color of a curve in a plot
%% \colorlet{penColor6}{yellow!70!black} % Color of a curve in a plot
%% \colorlet{fill1}{penColor!20} % Color of fill in a plot
%% \colorlet{fill2}{penColor2!20} % Color of fill in a plot
%% \colorlet{fillp}{fill1} % Color of positive area
%% \colorlet{filln}{penColor2!20} % Color of negative area
%% \colorlet{fill3}{penColor3!20} % Fill
%% \colorlet{fill4}{penColor4!20} % Fill
%% \colorlet{fill5}{penColor5!20} % Fill
%% \colorlet{gridColor}{gray!50} % Color of grid in a plot

%% \newcommand{\surfaceColor}{violet}
%% \newcommand{\surfaceColorTwo}{redyellow}
%% \newcommand{\sliceColor}{greenyellow}




%% \pgfmathdeclarefunction{gauss}{2}{% gives gaussian
%%   \pgfmathparse{1/(#2*sqrt(2*pi))*exp(-((x-#1)^2)/(2*#2^2))}%
%% }





%% %\typeout{************************************************}
%% %\typeout{ORCCA Preamble.Tex}
%% %\typeout{************************************************}


%% %% \usepackage{geometry}
%% %% \geometry{letterpaper,total={408pt,9.0in}}
%% %% Custom Page Layout Adjustments (use latex.geometry)
%% %% \usepackage{amsmath,amssymb}
%% %% \usepackage{pgfplots}
%% \usepackage{pifont}                                         %needed for symbols, s.a. airplane symbol
%% \usetikzlibrary{positioning,fit,backgrounds}                %needed for nested diagrams
%% \usetikzlibrary{calc,trees,positioning,arrows,fit,shapes}   %needed for set diagrams
%% \usetikzlibrary{decorations.text}                           %needed for text following a curve
%% \usetikzlibrary{arrows,arrows.meta}                         %needed for open/closed intervals
%% \usetikzlibrary{positioning,3d,shapes.geometric}            %needed for 3d number sets tower
%% % \usetkzobj{all}       %NO LONGER VALID
%% \usepackage{tikz-3dplot}
%% \usepackage{tkz-euclide}                     %needed for triangle diagrams
%% \usepgfplotslibrary{fillbetween}                            %shade regions of a plot
%% \usetikzlibrary{shadows}                                    %function diagrams
%% \usetikzlibrary{positioning}                                %function diagrams
%% \usetikzlibrary{shapes}                                     %function diagrams
%% %%% global colors from https://www.pcc.edu/web-services/style-guide/basics/color/ %%%
%% \definecolor{ruby}{HTML}{9e0c0f}
%% \definecolor{turquoise}{HTML}{008099}
%% \definecolor{emerald}{HTML}{1c8464}
%% \definecolor{amber}{HTML}{c7502a}
%% \definecolor{amethyst}{HTML}{70485b}
%% \definecolor{sapphire}{HTML}{263c53}
%% \colorlet{firstcolor}{sapphire}
%% \colorlet{secondcolor}{turquoise}
%% \colorlet{thirdcolor}{emerald}
%% \colorlet{fourthcolor}{amber}
%% \colorlet{fifthcolor}{amethyst}
%% \colorlet{sixthcolor}{ruby}
%% \colorlet{highlightcolor}{green!50!black}
%% \colorlet{graphbackground}{white}
%% \colorlet{wood}{brown!60!white}
%% %%% curve, dot, and graph custom styles %%%
%% \pgfplotsset{firstcurve/.style      = {color=firstcolor,  mark=none, line width=1pt, {Kite}-{Kite}, solid}}
%% \pgfplotsset{secondcurve/.style     = {color=secondcolor, mark=none, line width=1pt, {Kite}-{Kite}, solid}}
%% \pgfplotsset{thirdcurve/.style      = {color=thirdcolor,  mark=none, line width=1pt, {Kite}-{Kite}, solid}}
%% \pgfplotsset{fourthcurve/.style     = {color=fourthcolor, mark=none, line width=1pt, {Kite}-{Kite}, solid}}
%% \pgfplotsset{fifthcurve/.style      = {color=fifthcolor,  mark=none, line width=1pt, {Kite}-{Kite}, solid}}
%% \pgfplotsset{highlightcurve/.style  = {color=highlightcolor,  mark=none, line width=5pt, -, opacity=0.3}}   % thick, opaque curve for highlighting
%% \pgfplotsset{asymptote/.style       = {color=gray, mark=none, line width=1pt, <->, dashed}}
%% \pgfplotsset{symmetryaxis/.style    = {color=gray, mark=none, line width=1pt, <->, dashed}}
%% \pgfplotsset{guideline/.style       = {color=gray, mark=none, line width=1pt, -}}
%% \tikzset{guideline/.style           = {color=gray, mark=none, line width=1pt, -}}
%% \pgfplotsset{altitude/.style        = {dashed, color=gray, thick, mark=none, -}}
%% \tikzset{altitude/.style            = {dashed, color=gray, thick, mark=none, -}}
%% \pgfplotsset{radius/.style          = {dashed, thick, mark=none, -}}
%% \tikzset{radius/.style              = {dashed, thick, mark=none, -}}
%% \pgfplotsset{rightangle/.style      = {color=gray, mark=none, -}}
%% \tikzset{rightangle/.style          = {color=gray, mark=none, -}}
%% \pgfplotsset{closedboundary/.style  = {color=black, mark=none, line width=1pt, {Kite}-{Kite},solid}}
%% \tikzset{closedboundary/.style      = {color=black, mark=none, line width=1pt, {Kite}-{Kite},solid}}
%% \pgfplotsset{openboundary/.style    = {color=black, mark=none, line width=1pt, {Kite}-{Kite},dashed}}
%% \tikzset{openboundary/.style        = {color=black, mark=none, line width=1pt, {Kite}-{Kite},dashed}}
%% \tikzset{verticallinetest/.style    = {color=gray, mark=none, line width=1pt, <->,dashed}}
%% \pgfplotsset{soliddot/.style        = {color=firstcolor,  mark=*, only marks}}
%% \pgfplotsset{hollowdot/.style       = {color=firstcolor,  mark=*, only marks, fill=graphbackground}}
%% \pgfplotsset{blankgraph/.style      = {xmin=-10, xmax=10,
%%                                         ymin=-10, ymax=10,
%%                                         axis line style={-, draw opacity=0 },
%%                                         axis lines=box,
%%                                         major tick length=0mm,
%%                                         xtick={-10,-9,...,10},
%%                                         ytick={-10,-9,...,10},
%%                                         grid=major,
%%                                         grid style={solid,gray!20},
%%                                         xticklabels={,,},
%%                                         yticklabels={,,},
%%                                         minor xtick=,
%%                                         minor ytick=,
%%                                         xlabel={},ylabel={},
%%                                         width=0.75\textwidth,
%%                                       }
%%             }
%% \pgfplotsset{numberline/.style      = {xmin=-10,xmax=10,
%%                                         minor xtick={-11,-10,...,11},
%%                                         xtick={-10,-5,...,10},
%%                                         every tick/.append style={thick},
%%                                         axis y line=none,
%%                                         y=15pt,
%%                                         axis lines=middle,
%%                                         enlarge x limits,
%%                                         grid=none,
%%                                         clip=false,
%%                                         axis background/.style={},
%%                                         after end axis/.code={
%%                                           \path (axis cs:0,0)
%%                                           node [anchor=north,yshift=-0.075cm] {\footnotesize 0};
%%                                         },
%%                                         every axis x label/.style={at={(current axis.right of origin)},anchor=north},
%%                                       }
%%             }
%% \pgfplotsset{openinterval/.style={color=firstcolor,mark=none,ultra thick,{Parenthesis}-{Parenthesis}}}
%% \pgfplotsset{openclosedinterval/.style={color=firstcolor,mark=none,ultra thick,{Parenthesis}-{Bracket}}}
%% \pgfplotsset{closedinterval/.style={color=firstcolor,mark=none,ultra thick,{Bracket}-{Bracket}}}
%% \pgfplotsset{closedopeninterval/.style={color=firstcolor,mark=none,ultra thick,{Bracket}-{Parenthesis}}}
%% \pgfplotsset{infiniteopeninterval/.style={color=firstcolor,mark=none,ultra thick,{Kite}-{Parenthesis}}}
%% \pgfplotsset{openinfiniteinterval/.style={color=firstcolor,mark=none,ultra thick,{Parenthesis}-{Kite}}}
%% \pgfplotsset{infiniteclosedinterval/.style={color=firstcolor,mark=none,ultra thick,{Kite}-{Bracket}}}
%% \pgfplotsset{closedinfiniteinterval/.style={color=firstcolor,mark=none,ultra thick,{Bracket}-{Kite}}}
%% \pgfplotsset{infiniteinterval/.style={color=firstcolor,mark=none,ultra thick,{Kite}-{Kite}}}
%% \pgfplotsset{interval/.style= {ultra thick, -}}
%% %%% cycle list of plot styles for graphs with multiple plots %%%
%% \pgfplotscreateplotcyclelist{pccstylelist}{%
%%   firstcurve\\%
%%   secondcurve\\%
%%   thirdcurve\\%
%%   fourthcurve\\%
%%   fifthcurve\\%
%% }
%% %%% default plot settings %%%
%% \pgfplotsset{every axis/.append style={
%%   axis x line=middle,    % put the x axis in the middle
%%   axis y line=middle,    % put the y axis in the middle
%%   axis line style={<->}, % arrows on the axis
%%   scaled ticks=false,
%%   tick label style={/pgf/number format/fixed},
%%   xlabel={$x$},          % default put x on x-axis
%%   ylabel={$y$},          % default put y on y-axis
%%   xmin = -7,xmax = 7,    % most graphs have this window
%%   ymin = -7,ymax = 7,    % most graphs have this window
%%   domain = -7:7,
%%   xtick = {-6,-4,...,6}, % label these ticks
%%   ytick = {-6,-4,...,6}, % label these ticks
%%   yticklabel style={inner sep=0.333ex},
%%   minor xtick = {-7,-6,...,7}, % include these ticks, some without label
%%   minor ytick = {-7,-6,...,7}, % include these ticks, some without label
%%   scale only axis,       % don't consider axis and tick labels for width and height calculation
%%   cycle list name=pccstylelist,
%%   tick label style={font=\footnotesize},
%%   legend cell align=left,
%%   grid = both,
%%   grid style = {solid,gray!20},
%%   axis background/.style={fill=graphbackground},
%% }}
%% \pgfplotsset{framed/.style={axis background/.style ={draw=gray}}}
%% %\pgfplotsset{framed/.style={axis background/.style ={draw=gray,fill=graphbackground,rounded corners=3ex}}}
%% %%% other tikz (not pgfplots) settings %%%
%% %\tikzset{axisnode/.style={font=\scriptsize,text=black}}
%% \tikzset{>=stealth}
%% %%% for nested diagram in types of numbers section %%%
%% \newcommand\drawnestedsets[4]{
%%   \def\position{#1}             % initial position
%%   \def\nbsets{#2}               % number of sets
%%   \def\listofnestedsets{#3}     % list of sets
%%   \def\reversedlistofcolors{#4} % reversed list of colors
%%   % position and draw labels of sets
%%   \coordinate (circle-0) at (#1);
%%   \coordinate (set-0) at (#1);
%%   \foreach \set [count=\c] in \listofnestedsets {
%%     \pgfmathtruncatemacro{\cminusone}{\c - 1}
%%     % label of current set (below previous nested set)
%%     \node[below=3pt of circle-\cminusone,inner sep=0]
%%     (set-\c) {\set};
%%     % current set (fit current label and previous set)
%%     \node[circle,inner sep=0,fit=(circle-\cminusone)(set-\c)]
%%     (circle-\c) {};
%%   }
%%   % draw and fill sets in reverse order
%%   \begin{scope}[on background layer]
%%     \foreach \col[count=\c] in \reversedlistofcolors {
%%       \pgfmathtruncatemacro{\invc}{\nbsets-\c}
%%       \pgfmathtruncatemacro{\invcplusone}{\invc+1}
%%       \node[circle,draw,fill=\col,inner sep=0,
%%       fit=(circle-\invc)(set-\invcplusone)] {};
%%     }
%%   \end{scope}
%%   }
%% \ifdefined\tikzset
%% \tikzset{ampersand replacement = \amp}
%% \fi
%% \newcommand{\abs}[1]{\left\lvert#1\right\rvert}
%% %\newcommand{\point}[2]{\left(#1,#2\right)}
%% \newcommand{\highlight}[1]{\definecolor{sapphire}{RGB}{59,90,125} {\color{sapphire}{{#1}}}}
%% \newcommand{\firsthighlight}[1]{\definecolor{sapphire}{RGB}{59,90,125} {\color{sapphire}{{#1}}}}
%% \newcommand{\secondhighlight}[1]{\definecolor{emerald}{RGB}{20,97,75} {\color{emerald}{{#1}}}}
%% \newcommand{\unhighlight}[1]{{\color{black}{{#1}}}}
%% \newcommand{\lowlight}[1]{{\color{lightgray}{#1}}}
%% \newcommand{\attention}[1]{\mathord{\overset{\downarrow}{#1}}}
%% \newcommand{\nextoperation}[1]{\mathord{\boxed{#1}}}
%% \newcommand{\substitute}[1]{{\color{blue}{{#1}}}}
%% \newcommand{\pinover}[2]{\overset{\overset{\mathrm{\ #2\ }}{|}}{\strut #1 \strut}}
%% \newcommand{\addright}[1]{{\color{blue}{{{}+#1}}}}
%% \newcommand{\addleft}[1]{{\color{blue}{{#1+{}}}}}
%% \newcommand{\subtractright}[1]{{\color{blue}{{{}-#1}}}}
%% \newcommand{\multiplyright}[2][\cdot]{{\color{blue}{{{}#1#2}}}}
%% \newcommand{\multiplyleft}[2][\cdot]{{\color{blue}{{#2#1{}}}}}
%% \newcommand{\divideunder}[2]{\frac{#1}{{\color{blue}{{#2}}}}}
%% \newcommand{\divideright}[1]{{\color{blue}{{{}\div#1}}}}
%% \newcommand{\negate}[1]{{\color{blue}{{-}}}\left(#1\right)}
%% \newcommand{\cancelhighlight}[1]{\definecolor{sapphire}{RGB}{59,90,125}{\color{sapphire}{{\cancel{#1}}}}}
%% \newcommand{\secondcancelhighlight}[1]{\definecolor{emerald}{RGB}{20,97,75}{\color{emerald}{{\bcancel{#1}}}}}
%% \newcommand{\thirdcancelhighlight}[1]{\definecolor{amethyst}{HTML}{70485b}{\color{amethyst}{{\xcancel{#1}}}}}
%% \newcommand{\lt}{<} %% Bart: WHY?
%% \newcommand{\gt}{>} %% Bart: WHY?
%% \newcommand{\amp}{&} %% Bart: WHY?


%% %%% These commands break Xake
%% %% \newcommand{\apple}{\text{🍎}}
%% %% \newcommand{\banana}{\text{🍌}}
%% %% \newcommand{\pear}{\text{🍐}}
%% %% \newcommand{\cat}{\text{🐱}}
%% %% \newcommand{\dog}{\text{🐶}}

%% \newcommand{\apple}{PICTURE OF APPLE}
%% \newcommand{\banana}{PICTURE OF BANANA}
%% \newcommand{\pear}{PICTURE OF PEAR}
%% \newcommand{\cat}{PICTURE OF CAT}
%% \newcommand{\dog}{PICTURE OF DOG}


%% %%%%% INDEX STUFF

\newcommand{\dfn}[1]{\textbf{#1}\index{#1}}
\usepackage{imakeidx}
\makeindex[intoc]
\makeatletter
\gdef\ttl@savemark{\sectionmark{}}
\makeatother
