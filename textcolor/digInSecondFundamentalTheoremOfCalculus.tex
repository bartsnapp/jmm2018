\documentclass{ximera}
\title[Dig-In:]{The Second Fundamental Theorem of Calculus}

\begin{document}
\begin{abstract}
The accumulation of a rate is given by the change in the amount.
\end{abstract}
\maketitle

\begin{quote}\large\textbf{The \textcolor{green!70!black!70!blue}{accumulation} of a \textcolor{blue!70!green}{rate} is given by the \textcolor{purple!50!blue!90!black}{change in the amount BOOM}.}
\end{quote}
% 
\begin{quote}\large\textbf{The accumulation of a rate is given by the change in the amount.}
\end{quote}

Arr--back?

Now to be ``continuous.'' We begin with a series of definitions. We are
very used to ``open intervals'' such as $(1,3)$, which represents the set
of all $x$ such that $1<x<3$, and ``closed intervals'' such as
$[1,3]$, which represents the set of all $x$ such that $1\leq x\leq
3$. 


\end{document}
